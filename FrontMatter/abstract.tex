\thispagestyle{empty}

\section*{\centering Abstract}

\begin{center}
    \large \textbf{Statistical and Machine Learning Methods for Pattern Identification in Environmental Mixtures} \\
    Elizabeth Atkeson Gibson
\end{center}

Abstract of dissertation. In the abstract, you must (1) present the problem of the thesis/dissertation, (2) discuss the materials and methods used, and (3) state the conclusions reached. Individual chapters should not have abstracts.	Abstract of dissertation. In the abstract, you must (1) present the problem of the thesis/dissertation, (2) discuss the materials and methods used, and (3) state the conclusions reached. Individual chapters should not have abstracts.
	
\paragraph{Purpose of Review.}

The purpose of this review is to outline the main questions in environmental mixtures research and provide a non-technical explanation of novel or advanced methods to answer these questions.

\paragraph{Recent Findings.} 

Machine learning techniques are now being incorporated into high-dimensional mixture research to overcome issues with traditional methods. Though some methods perform well on specific tasks, no method consistently outperforms all others in complex mixture analyses, largely because different methods were developed to answer different research questions. We discuss four main questions in environmental mixtures research: 1) Are there specific exposure patterns in the study population? 2) Which are the toxic agents in the mixture? 3) Are mixture members acting synergistically? and 4) What is the overall effect of the mixture? 

\paragraph{Summary.} 

We emphasize the importance of robust methods and interpretable results over predictive accuracy. We encourage collaboration with computer scientists, data scientists, and biostatisticians in future mixtures methods development.