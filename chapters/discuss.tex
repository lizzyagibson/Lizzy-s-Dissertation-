\chapter{Conclusion}\label{sec:conclusion}
\clearpage
This dissertation aimed to adapt pattern recognition methods employed in other fields (e.g., computer vision and machine learning) to environmental health data. We modified existing methods to remove the researcher from pattern number selection or specification, to retain the interpretability of a parts-based representation of data, and to explicitly account for uncertainty in identified patterns. We further incorporated this uncertainty in a hierarchical health model to understand the relationships between identified patterns of EDC exposure and child cognitive development. In this final chapter, we discuss our findings and their implications in three main sections: in Section~\ref{sec:summarize} we summarize our findings and contributions and discuss their position within environmental mixtures research, in Section~\ref{sec:future} we discuss future research directions, and in Section~\ref{sec:ph} we conclude with the public health relevance and implications of our work.

\section{Our findings amid current landscape}\label{sec:summarize}

\subsection{Dissertation findings}\label{sec:findings}
additions from this work

\subsection{Environmental mixtures}\label{sec:mixtures}
state of the science, quick though 

\subsection{Pattern recognition}\label{sec:patrec}
other methods, similar / different, similar / different from each other, why this is hard

\subsection{Bayesian methods}\label{sec:bayes}

%%%%%%%%%%%%%%%%%%
\clearpage
\section{Future research directions}\label{sec:future}
To date, environmental epidemiological investigations of multi-pollutant exposures have asked a variety of research questions and developed and adapted numerous statistical and machine learning methods to address them. However, a number of important questions remain unaddressed and new challenges have become apparent. Below, we outline several challenges and opportunities for future research on environmental mixtures and pattern identification therein.

\subsection{Future studies}
This dissertation detailed work to date to adapt two pattern identification methods for use in environmental mixtures research. Both tools, however, should be considered works in progress. We plan to incorporate various methodological extensions and to apply these methods to future research questions concerning mixtures in independent study populations.

For PCP-LOD we envision several elements that would improve performance in certain areas. First, as we emphasized in Chapter~\ref{sec:ch3}, when characterizing multi-pollutant exposures for use in health studies, and especially when the principal components cannot be directly measured, it is important to quantify the uncertainty in their estimation and propagate it in the health models. We propose, therefore, to extend PCP-LOD by coupling it with a bootstrap process to quantify and propagate uncertainty associated with the estimation of L and S \citep{mak14_unc}. Second, we aim to allow for repeated measures, e.g., children within schools or sequential measurements within a person, which will be correlated. As currently implemented, the Frobenius norm in the error term $\left( \|L+S-M\|_{F} \right)$ does not reflect this structure. Clustering information can be incorporated into a hypergraph, and error can be measured in a metric which reflects the smoothness of the noise with respect to the hypergraph---i.e., the Laplacian metric \citep{zhou2006learning, kaminski2019clustering}. At the same time, we aim to incorporate this clustering information into our estimation of S. For instance, since S captures rare events in the environment, it is possible that these events will occur in subjects living in close proximity. We will leverage ideas from hierarchical sparse modeling to this end, treating entries in S that belong to the same group as a group of variables, and penalizing both the sum of $\ell_2$ norms of the groups, as well as the $\ell_1$ norm within each group \citep{yan2017hierarchical, sprechmann2011c, jenatton2011proximal}.

For \bnmf we anticipate future work to reformulate the two-stage hierarchical health model as a fully supervised model where the outcome of interest informs the `grouping' of the chemicals in the mixture. As a Bayesian model, \bnmf can be naturally embedded in a more complex model to estimate pattern-specific health effects \citep{bda3}. Although such an approach would not necessarily be helpful in informing regulatory action, as different groupings could be formed depending on the outcome, it may provide some insight on the potential biological pathways, especially in exploratory analyses when biological plausibility has not yet been established. This structure should also serve to enhance reproducibility, as identified patterns based on biological mechanisms would be more similar across populations than those identified by the current iteration of \bnmfc.

The patterns found in Chapters~\ref{sec:ch2} and \ref{sec:ch3} correspond well with known sources, behaviors, toxicity and chemical structure. Nevertheless, we do not yet know whether identified patterns will generalize to other populations. The NHANES data used in Chapter~\ref{sec:ch2} represents the general non-institutionalized US population; however, we did not account for the complex sampling design and weights of the study \citep{johnson2013national}. Oversampling was performed for adolescents (aged 12–19), older Americans (aged 60 and over), Mexican-Americans, African-Americans, pregnant women, and individuals at or below 130\% of the poverty level \citep{curtin2012national}, and PCP-LOD-identified patterns may represent sources or behaviors distinct to included participants. Similarly, the Mothers and Newborns cohort studied in Chapters~\ref{sec:ch3} and \ref{sec:ch4} is composed of low-income Dominican and African American mothers living in an urban environment, and the \bnmfc-identified EDC patterns may represent sources or behaviors distinct to this group. To address the potential limitation of generalizability of our current findings, future studies will attempt to replicate results in independent study populations.

\subsection{Cross-disciplinary collaboration with toxicological research}
%%%%%%%%%%%%%%%%%%
tox studies -- take mixture, expose zebra fish to it, etc
%%%%%%%%%%%%%%%%%%

\subsection{Environmental mixtures of interest}
This work identified patterns underlying two environmental mixtures, one comprised of 21 PCBs, dioxins, and phenols measured in the 2001--2002 cycle of NHANES and one of 17 phenols and phthalate metabolites measured in pregnant women in New York City between 1998 and 2006. These are two of an uncountable number of conceivable chemical mixtures. The vast expanse of environmental exposures and their potential combinations present opportunities for future research on pattern recognition within an array of environmental mixtures. 

Both mixtures investigated in this work fall under the wider grouping of endocrine disrupting chemicals (EDCs) \citep{gore2015edc}. While PCBs, dioxins, and furans are perhaps better know for their cancer-causing properties, they also act as EDCs \citep{van1998toxic}, which may prove critical at the low exposure levels experienced post-regulation (PCB manufacturing was banned in 1979) \citep{trost1989regulation}. EDCs, generally, are structurally similar to endogenous human hormones and may mimic or interfere with normal hormonal processes \citep{kavlock1996research, zoeller2012endocrine}. Because of the complexity of the endocrine system, which is comprised of multiple interrelated feedback loops \citep{diamanti2009endocrine}, low level exposures may prove more detrimental than high level exposures, i.e., the dose-response curve may be steepest at low exposure levels. Expanding this logic to a mixture of EDCs, a pattern of generally high exposure may not be the most harmful. This makes research questions concerning patterns of EDC exposure especially relevant in future work. In addition to PCBs, phthalates, and phenols, EDCs of interest include perchlorate \citep{nizinski2020perchlorate}, perfluoroalkyl and polyfluoroalkyl substances (PFAS) \citep{pfaswebsite}, phyto-estrogens \citep{yilmaz2020endocrine}, and polybrominated diphenyl ethers (PBDE) \citep{gibson2018effects}. These chemicals are linked with developmental, reproductive, neurological, and immunological problems \citep{solomon2000environment, meeker2012exposure, kabir2015review}. The identification of underlying patterns of EDCs corresponding with actions or circumstances leading to exposure may inform interventions or regulatory action to curb exposure.

Another mixture of interest includes metals and metalloids, such as arsenic (As), cadmium (Cd), magnesium (Mn), mercury (Hg), and lead (Pb). A large body of literature exists linking environmental metal exposure with various health endpoints such as cardiovascular disease \citep{nigra2016environmental} and neurodevelopment \citep{henn2012associations}. In related work, we assessed the joint effects of coexposure to As, Cd, Mn, and Pb on adolescent intellectual function and blood pressure, and explored potential beneficial effects of selenium (Se), an element known to lessen the overt toxicity of As \citep{levander1977metabolic}. We observed significantly negative associations between As and Cd and general intellectual ability (while accounting for other metals) as well as a significant decrease in Full Scale intelligence quotient with increased exposure to the entire metals mixture \citep{wasserman2018cross}. We further observed significantly positive associations between As and Mn and systolic blood pressure (SBP) and between As, Se, Pb and Cd and diastolic blood pressure (DBP) as well as a significant increase in DBP with increased exposure to the metals mixture \citep{chen2019early}. Previous work, however, did not investigate potential patterns of metals exposure.

Air pollution is an additional multi-pollutant exposure of interest. Research in source apportionment to derive information about ambient pollution sources and the amount they contribute to air pollution levels is more mature than the environmental mixtures field as a whole \citep{paatero94, sun2020positive}. Nevertheless, there remain avenues for future work. For example, uncertainty propagation continues to prove challenging. While existing research demonstrates the use of various matrix factorization and clustering algorithms to identify patterns or profiles within air pollution, the current norm is to include patterns or clusters as identified in health models \citep{austin2012framework, zanobetti2014health, sarnat2008fine, krall2017associations}. In a previous study using source apportionment to assess air pollution source-specific impacts on cardiovascular admissions, \citet{mak14_unc} showed that failure to account for this uncertainty resulted in overestimated confidence in inference, potentially spurious findings, and disagreement across methods.

Of course, mixtures research need not be limited to chemicals. We can include markers of socio-economic status, demographic factors, nutrition, or built environment variable, to name a few. We can also conceptualize exposures from a more `exposomic' perspective which captures a diverse range of elements in addition to environmental chemicals, like dietary constituents, psychosocial stressors, lifestyle, and physical factors, as well as their corresponding biological responses (e.g., in the form of metabolites, hormones, or proteins) \citep{vermeulen2020exposome}.

% PAHs
% pesticides & OPFR

\subsection{Reproducibility in science}
There has been substantial discourse within the scientific community concerning the reproducibility of research. If a different researcher cannot run the same analysis in the future with the same setup and get the same result, is the original work good science \citep{goodman2016does}? Reproducibility in epidemiology is made more complex by the ethical matter of personally identifiable information (PII), however, availability of the analytic code, as a minimum, allows other researchers to repeat the analysis on independently collected data \citep{peng2011reproducible}. Replication of results in a separate study population strengthens evidence \citep{peng2006reproducible}.

Packaging code into reusable analysis software enables the dissemination of tools to facilitate future work more efficaciously. This is a good step in establishing reproducibility. Subject to the ethical constraints of PII, researchers should endeavor to make computations reproducible for others and future versions of themselves \citep{barnes2010publish}. Analytic code and simulated data accompanying this work are publicly available on GitHub in three repositories: Chapter~\ref{sec:ch2} in \texttt{\href{https://github.com/lizzyagibson/Principal.Component.Pursuit}{Principal.Component.Pursuit}}, Chapter~\ref{sec:ch3} in \texttt{\href{https://github.com/lizzyagibson/BN2MF}{BN2MF}}, and Chapter~\ref{sec:ch4} in \texttt{\href{https://github.com/lizzyagibson/edc-patterns-wisc}{edc-patterns-wisc}}. We included the cleaned NHANES dataset with the code from Chapter~\ref{sec:ch2} because it is publicly available data. We did not include the Mothers and Newborns cohort data accompanying analyses in Chapters~\ref{sec:ch3} and \ref{sec:ch4}. These data were collected with written informed consent of mothers (and assent of children at age seven) \citep{perera03}; this did not include consent to publication of their personal data.

Further, we found in related work that details such as package version and software environment (e.g., R version) affect reproducibility of results in epidemiological studies \citep{nunez2020reflection}. Upgrading a platform or moving between computer environments should not obstruct replication or reproduction of results. Packaging and version control enable consistent inputs to produce consistent outputs \citep{wilson2014best}, which allows validation of analytic code by a third party.

\subsection{Accessibility of methods}
Completion of the preceding chapters required several classes in linear algebra, sparse and low dimensional modelling, and probabilistic machine learning. Implementation of PCP-LOD and \bnmf employed two complex optimization algorithms. However, we developed both methods with the end user in mind. We took steps to remove the researcher from specification or selection of pattern number. We incorporated universal values for hyper-parameters in PCP-LOD so that users would not need to tune them. We chose vague priors for individual scores and chemical loadings in \bnmf so that they would suit various configurations of environmental data. We chose to approximate \bnmfc's posterior distribution through variational inference in part because it lifted the burden of assessing model convergence from the researcher. We mean for both methods to be broadly accessible to environmental health scientists and epidemiologists, who may lack formal statistical or mathematical training. Thus, as future work, we plan to develop and share user-friendly statistical software packages so that other researchers assessing exposures to mixtures can easily apply PCP-LOD or \bnmf to their research.

We propose to develop separate R packages for PCP-LOD (and other PCP implementations) and \bnmf to facilitate their use in environmental epidemiologic applications. These will provide environmental epidemiologists with accessible and flexible tools to address study-specific needs. We will include in the packaging all developed synthetic datasets as examples, along with documentation for guidance on proper use, inputs, interpretation, and limitations of both methods. We believe that the existence of user-friendly packages, such as \texttt{bkmr}, \texttt{gWQS}, and \texttt{qgcomp}, encourages researchers to incorporate complex methodologies by making them more approachable \citep{bobb2018statistical, renzetti2016gwqs, keil2020quantile}.

\section{Public Health relevance}\label{sec:ph}
\subsection{Research questions}\label{sec:question}
A unifying theme in this work is the importance of the research question in driving the choice of statistical or machine learning method. For example, \bnmf cannot distinguish important individual mixture members or `bad actors,' and linear regression cannot identify underlying patterns of exposure. Further, the research question should be informed by the ultimate goal of the project---e.g., is it to provide evidence of a plausible biological pathway between chemical and disease or to support comprehensive regulation of chemicals that co-occur? In this dissertation, we introduced PCP-LOD and \bnmf as methods to better understand the underlying structure of environmental exposures.

Both are pattern recognition methods designed to address environmental health research questions concerning patterns of environmental exposures and to identify potential sources or behaviors leading to these exposures. The majority of environmental exposures are modifiable, meaning that individual action or, more sustainably, regulatory action can prevent or (at least) reduce exposure. Accordingly, the fundamental aim of this work is to design methods capable of establishing actions or circumstances that contribute to simultaneous chemical exposures in support of targeted public health interventions and regulations. In Chapter~\ref{sec:ch4} we observed a negative association between a pattern of prenatal phthalate and BPA exposure and female child IQ at seven years of age. This finding supports public health and regulative action on EDCs used as plasticizers and additives in food packaging; it does not support action against a single phthalate or a single phenol. Thus, this research supports a shift from chemical-by-chemical regulation toward \textit{class-based regulation}, where groups of chemicals identified as sharing properties and risks are evaluated and regulated together \citep{cordner2016can}.

\subsection{Public Health implications}\label{sec:fin}
No individual is exposed to a single chemical at a time. While public health research has traditionally relied upon well-loved standards such as linear and logistic regression to determine population-level associations between chemical exposures and outcomes, some research questions concerning multi-pollutant exposures require novel methods. The current focus on chemical mixtures represents a critical juncture in environmental health research. With this perspective, in Chapters~\ref{sec:ch2} and \ref{sec:ch3} we identified chemical patterns that we interpreted as sources of exposure within environmental mixtures; recognition of these patterns may aid in the development of preventive strategies to minimize exposure in individuals.

Researchers may use both PCP-LOD and \bnmf to provide foundational evidence for pattern-based regulatory action or informed interventions, where patterns may represent exposure sourcse (e.g., diet as a source of POPs in Chapter~\ref{sec:ch2} and phthalates and BPA in Chapter~\ref{sec:ch3} and  \ref{sec:ch4}), at-risk behaviors (e.g., personal care product use contributing to phenol and DEP exposure in Chapters~\ref{sec:ch3} and \ref{sec:ch4}), or similar chemical structure (e.g., high and low molecular weight PCBs in Chapter~\ref{sec:ch2}). Both methods can be paired with expert knowledge to identify modifiable risk factors which are more effectively targeted than individual chemicals; for example, a source of particulate matter (PM), such as traffic, is more easily regulated than a single PM component, such as copper.

Depending on the health outcome, implementation of strategies to reduce or prevent exposure to contributing environmental factors could have a consequential impact. In Chapter~\ref{sec:ch4} we considered the relationship between \textit{in utero} EDC exposure and child cognition. This work supports targeted public health interventions with expectant mothers or women contemplating pregnancy concerning their dietary choices. More importantly, confirmed associations between exposure patterns and neurodevelopment provide leverage to demand stricter governmental regulations of food packaging and actions to curb exposure.

Environmental health scientists and epidemiologists can use PCP-LOD and \bnmf to investigate shared sources of chemical exposure and behaviors and circumstances leading to exposure. We adapted these methods to suit environmental data (i.e., chemical concentrations) and to answer a subset of environmental mixtures questions. Research on underlying patterns of chemical exposure and unique or extreme exposure events can aid in the design and development of class-based regulations, informed policies, and targeted interventions to ensure equal access to a clean environment.