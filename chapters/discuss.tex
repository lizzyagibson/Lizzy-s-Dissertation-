\chapter{Conclusion}\label{sec:conclusion}
\clearpage

This dissertation aimed to adapt pattern recognition methods employed in other fields (i.e., computer vision and machine learning) to environmental health data. We modified existing methods to remove the researcher from pattern number selection or specification, to retain the interpretability of a parts-based representation of data, and to explicitly account for uncertainty in identified patterns. We further incorporated this uncertainty in a hierarchical health model to understand the relationships between identified patterns of EDC exposure and child cognitive development. In this final chapter, we discuss our findings and their implications in three main sections: in Section~\ref{sec:summarize} we summarize our results and discuss their position within environmental mixtures research, in Section~\ref{sec:future} we discuss future research directions, and in Section~\ref{sec:ph} we conclude with the public health relevance and implications of our work.

\section{Findings \& environmental mixtures research}\label{sec:summarize}

\subsection{Dissertation findings}\label{sec:findings}
additions from this work

\subsection{Environmental mixtures}\label{sec:mixtures}
state of the science \\
examples, source apportionment

\subsection{Pattern recognition}\label{sec:patrec}
other methods \\
similar / different \\

\subsection{Bayesian methods}\label{sec:bayes}

\section{Future research directions}\label{sec:future}

\begin{itemize}
    \item tox studies
    \item coupling pattern recognition with other methods
    \item uncertainty propagation
    \item other mixtures
\end{itemize}

\section{Public Health relevance}\label{sec:ph}

\subsection{Research questions}\label{sec:question}
what questions should we be asking \\
research question should be informed by ultimate goal \\
what questions can we answer with this?

\subsection{Public Health implications}\label{sec:fin}

\begin{itemize}
    \item policy implications / informed interventions
    \item generalizability of patterns and of MN cohort
    \item biological mechanism not really with this method
\end{itemize}