\setabstract{
Abstract of dissertation. In the abstract, you must (1) present the problem of the thesis/dissertation, (2) discuss the materials and methods used, and (3) state the conclusions reached. Individual chapters should not have abstracts.	Abstract of dissertation. In the abstract, you must (1) present the problem of the thesis/dissertation, (2) discuss the materials and methods used, and (3) state the conclusions reached. Individual chapters should not have abstracts.
	
\textbf{Intro abstract}
\paragraph{Purpose of Review.}

The purpose of this review is to outline the main questions in environmental mixtures research and provide a non-technical explanation of novel or advanced methods to answer these questions.

\paragraph{Recent Findings.} 

Machine learning techniques are now being incorporated into high-dimensional mixture research to overcome issues with traditional methods. Though some methods perform well on specific tasks, no method consistently outperforms all others in complex mixture analyses, largely because different methods were developed to answer different research questions. We discuss four main questions in environmental mixtures research: 1) Are there specific exposure patterns in the study population? 2) Which are the toxic agents in the mixture? 3) Are mixture members acting synergistically? and 4) What is the overall effect of the mixture? 

\paragraph{Summary.} 

We emphasize the importance of robust methods and interpretable results over predictive accuracy. We encourage collaboration with computer scientists, data scientists, and biostatisticians in future mixtures methods development.

\textbf{WISC abstract}

A growing body of evidence links prenatal endocrine disrupting chemical (EDC) exposure with adverse cognitive development in children. Further research addresses EDCs as an environmental mixture. Here, we investigate the relationship between EDC exposure patterns in pregnant women and their children's cognitive development.

We measured exposure to 17 phenols and phthalates in 343 pregnant women in a mother-child cohort and full scale intelligence quotient (IQ) in children at seven years of age. We designed a two-stage Bayesian hierarchical model to estimate health effects of environmental exposure patterns while incorporating the uncertainty of pattern identification. In the first stage, we identified EDC exposure patterns using Bayesian non-parametric non-negative matrix factorization (\bnmfc). In the second stage, we included individual pattern scores and their distributions as exposures of interest in a linear regression model, with child IQ as the outcome, adjusting for potential confounders. We present sex-specific results.

The two \bnmfc-identified patterns of EDC exposure in pregnant mothers corresponded with diet and personal care product use as potentially separate sources or behaviors leading to exposure. The diet pattern expressed exposure to higher molecular weight phthalates and BPA. One standard deviation increase in this pattern was associated with a decrease of 3.5 IQ points (95\% credible interval: -6.7, -0.3), on average, in female children but not in males. The personal care product pattern represented exposure to phenols, including parabens, and di-ethyl phthalate. We found no associations between this pattern and child cognition.

Phthalates and BPA found in food packaging and can linings formed a pattern of EDC exposure negatively associated with female child intelligence. Results may be used to inform interventions designed to target modifiable behavior or regulations to act on dietary exposure sources.

\textbf{bn2mf}
Environmental health researchers may aim to identify exposure patterns that represent sources, product use, or behaviors that give rise to mixtures of potentially harmful environmental chemical exposures. Existing pattern recognition methods employed in this field are limited by user-specified pattern number, lack of interpretability of patterns in terms of human understanding, and lack of uncertainty quantification. We adapted Bayesian non-parametric non-negative matrix factorization (\bnmfc) to identify patterns of chemical exposures when the number of patterns is not known \textit{a priori}. We placed non-negative continuous priors on pattern loadings and individual scores and used a non-parametric sparse prior to estimate the pattern number. To test this method, we simulated 12,100 datasets with increasing levels of complexity in pattern structure and increasingly noisy scenarios. We assessed \bnmfc's coverage of true simulated scores and compared our model's performance with methods frequently used in environmental health. \bnmf estimated the true number of patterns for 99\% of simulated datasets. \bnmfc's variational confidence intervals achieved 95\% coverage across all levels of structural complexity with up to 40\% added noise. \bnmf performed comparably with frequentist methods in terms of overall prediction and estimation of underlying loadings and scores.

After validation, we applied \bnmf to a mixture of 17 potentially endocrine disrupting chemicals (EDCs) measured in 343 pregnant women in the Columbia Center for Children’s Environmental Health's Mothers and Newborns Cohort.  We identified two patterns of EDC exposure in pregnant women, corresponding well with known EDC sources. \bnmf can be a useful tool to identify patterns in environmental mixtures.
} % this closes abstract environment

